\documentclass[]{book}
\usepackage{lmodern}
\usepackage{amssymb,amsmath}
\usepackage{ifxetex,ifluatex}
\usepackage{fixltx2e} % provides \textsubscript
\ifnum 0\ifxetex 1\fi\ifluatex 1\fi=0 % if pdftex
  \usepackage[T1]{fontenc}
  \usepackage[utf8]{inputenc}
\else % if luatex or xelatex
  \ifxetex
    \usepackage{mathspec}
  \else
    \usepackage{fontspec}
  \fi
  \defaultfontfeatures{Ligatures=TeX,Scale=MatchLowercase}
\fi
% use upquote if available, for straight quotes in verbatim environments
\IfFileExists{upquote.sty}{\usepackage{upquote}}{}
% use microtype if available
\IfFileExists{microtype.sty}{%
\usepackage{microtype}
\UseMicrotypeSet[protrusion]{basicmath} % disable protrusion for tt fonts
}{}
\usepackage[margin=1in]{geometry}
\usepackage{hyperref}
\hypersetup{unicode=true,
            pdftitle={EMM-JRC congress notes},
            pdfauthor={Corrado Lanera},
            pdfborder={0 0 0},
            breaklinks=true}
\urlstyle{same}  % don't use monospace font for urls
\usepackage{natbib}
\bibliographystyle{apalike}
\usepackage{longtable,booktabs}
\usepackage{graphicx,grffile}
\makeatletter
\def\maxwidth{\ifdim\Gin@nat@width>\linewidth\linewidth\else\Gin@nat@width\fi}
\def\maxheight{\ifdim\Gin@nat@height>\textheight\textheight\else\Gin@nat@height\fi}
\makeatother
% Scale images if necessary, so that they will not overflow the page
% margins by default, and it is still possible to overwrite the defaults
% using explicit options in \includegraphics[width, height, ...]{}
\setkeys{Gin}{width=\maxwidth,height=\maxheight,keepaspectratio}
\IfFileExists{parskip.sty}{%
\usepackage{parskip}
}{% else
\setlength{\parindent}{0pt}
\setlength{\parskip}{6pt plus 2pt minus 1pt}
}
\setlength{\emergencystretch}{3em}  % prevent overfull lines
\providecommand{\tightlist}{%
  \setlength{\itemsep}{0pt}\setlength{\parskip}{0pt}}
\setcounter{secnumdepth}{5}
% Redefines (sub)paragraphs to behave more like sections
\ifx\paragraph\undefined\else
\let\oldparagraph\paragraph
\renewcommand{\paragraph}[1]{\oldparagraph{#1}\mbox{}}
\fi
\ifx\subparagraph\undefined\else
\let\oldsubparagraph\subparagraph
\renewcommand{\subparagraph}[1]{\oldsubparagraph{#1}\mbox{}}
\fi

%%% Use protect on footnotes to avoid problems with footnotes in titles
\let\rmarkdownfootnote\footnote%
\def\footnote{\protect\rmarkdownfootnote}

%%% Change title format to be more compact
\usepackage{titling}

% Create subtitle command for use in maketitle
\newcommand{\subtitle}[1]{
  \posttitle{
    \begin{center}\large#1\end{center}
    }
}

\setlength{\droptitle}{-2em}
  \title{EMM-JRC congress notes}
  \pretitle{\vspace{\droptitle}\centering\huge}
  \posttitle{\par}
  \author{Corrado Lanera}
  \preauthor{\centering\large\emph}
  \postauthor{\par}
  \predate{\centering\large\emph}
  \postdate{\par}
  \date{2018-01-19}

\usepackage{booktabs}

\usepackage{amsthm}
\newtheorem{theorem}{Theorem}[chapter]
\newtheorem{lemma}{Lemma}[chapter]
\theoremstyle{definition}
\newtheorem{definition}{Definition}[chapter]
\newtheorem{corollary}{Corollary}[chapter]
\newtheorem{proposition}{Proposition}[chapter]
\theoremstyle{definition}
\newtheorem{example}{Example}[chapter]
\theoremstyle{definition}
\newtheorem{exercise}{Exercise}[chapter]
\theoremstyle{remark}
\newtheorem*{remark}{Remark}
\newtheorem*{solution}{Solution}
\begin{document}
\maketitle

{
\setcounter{tocdepth}{1}
\tableofcontents
}
\chapter*{Introduction}\label{introduction}
\addcontentsline{toc}{chapter}{Introduction}

These notes are about the workshop on the Europe Media Monitoring tool
as presented in Gazzada in occasion of the EMM congress provided by the
Text Mining Team of the JRC (Joint Research Center).

The congress span on three days (29-30 November and 1 December 2017) and
it is divided in two main part:

\begin{enumerate}
\def\labelenumi{\arabic{enumi}.}
\tightlist
\item
  Mornings: conference in which many application and characteristics of
  EMM has been communicated.
\item
  Afternoons: Workshops, which are three parallel ones on the usability
  on:

  \begin{itemize}
  \tightlist
  \item
    overview of the system and the on-line platform,
  \item
    use of the desktop app and relative API,
  \item
    internal procedures used by EMM.
  \end{itemize}
\end{enumerate}

Organization of the notes: one chapter per part of the congress, each
one is subdivided in sub-chapter corresponding to the talks (in the
first chapter) or corresponding to the the topics of the workshop.

\chapter{Conference}\label{conf}

\section[International co-operation in Financial Criminal
Investigations]{\texorpdfstring{International co-operation in Financial
Criminal Investigations\footnote{\emph{Eugenie de Lange} --- Dutch Tax
  Authorities}}{International co-operation in Financial Criminal Investigations}}\label{international-co-operation-in-financial-criminal-investigations}

\begin{quote}
Abstract: Technological developments make things possible that we could
not have imagined twenty-five years ago. Forms of communication,
producing things, transporting goods, the way in which transactions take
place are changing constantly, not only in legal, but also in illegal
businesses. To effectively fight serious financial cross border
criminality, international cooperation is a necessity. The past year two
international initiatives were launched which had financial criminal
investigations as the starting point. ENFIN, a European knowledge
Network were members and partners can exchange experiences, methods and
new developments on Financial Investigations, and FCInet a virtual
solution that makes data matching and exchange easy and enlarge the
international cooperation possibilities. FCInet is based on the proven
technology and is supported by the Forum of Tax Crime Investigations
(FHTCI) of the OECD. Both networks consist of Law Enforcement
organisations specialised in financial investigations and Tax and
Customs administrations. They collaborate with respect to the
differences in working methods, (privacy) legislation, data protection
and independence of the participating organisations. The presentation
will contain backgrounds on the ENFIN and FCInet, the aims, partners,
steps taken and the outlook for 2018.
\end{quote}

\section[SIRIUS as a law frameworkfor internet
investigations]{\texorpdfstring{SIRIUS as a law frameworkfor internet
investigations\footnote{\emph{Juan De Toledo Maetinez} --- EUROPOL}}{SIRIUS as a law frameworkfor internet investigations}}\label{sirius-as-a-law-frameworkfor-internet-investigations}

\begin{quote}
Abstract: Nowadays, investigators working on a case cannot avoid
investigating the digital footprint of those who plan terrorist attacks
or are suspected of recruitment, training and financing of terrorism, as
well as incitement to commit a terrorist offense, including
relationships, communication means, financial aspects, logistics,
centres of interest and behavioural activity. Whilst these information
sets were historically under the remit of national entities, they have
now acquired a global perspective and are owned by Online Service
Providers, oftentimes based outside of the EU territory. Further
complexity stems both from the volatility of data held across different
legislations, as well as its volume and the urgency with which this
information is needed in the context of a CT case In the attempt to cope
with these challenges and to maximize the level and the quality of
operational support provided, we have recently launched SIRIUS, a
project which aims to cater for the investigators' needs in an online
environment. Available only to law enforcement authorities and deployed
in a closed and secured environment, SIRIUS is the place where all the
information related to Online Service Providers, and how to increase the
investigation efficiency can be found, with manuals, tips, forums, Q\&A,
etc. Additionally, the platform will also include a repository for the
collaborative development of tools to support investigations of crimes
facilitated by the Internet, developed by and intended for the Law
Enforcement community.
\end{quote}

\begin{itemize}
\tightlist
\item
  \url{https://www.europol.europa.eu/newsroom/news/europol-launches-sirius-platform-to-facilitate-online-investigations}
\end{itemize}

\begin{enumerate}
\def\labelenumi{\arabic{enumi}.}
\item
  26 attacks since January 2016
\item
  Communications:
\end{enumerate}

\begin{itemize}
\tightlist
\item
  many to many --- propaganda
\item
  one to many --- recruitment
\item
  one to one --- private communications
\end{itemize}

\begin{enumerate}
\def\labelenumi{\arabic{enumi}.}
\setcounter{enumi}{2}
\tightlist
\item
  Steps:
\end{enumerate}

\begin{itemize}
\tightlist
\item
  standardization of data collections
\item
  identification of nodes about the \emph{big plyers} do the (net)
\item
  high profiles investigation
\end{itemize}

\begin{enumerate}
\def\labelenumi{\arabic{enumi}.}
\setcounter{enumi}{3}
\tightlist
\item
  Big Data:
\end{enumerate}

\begin{itemize}
\tightlist
\item
  Open
\item
  Closed/Private
\end{itemize}

\begin{quote}
Europol Experts --\textgreater{} SIRIUS project: 263 members, 13 ONSIT
tools
\end{quote}

\section[Social Media Monitoring for Awareness of Security threats
against VIPs: opportunities and challenges]{\texorpdfstring{Social Media
Monitoring for Awareness of Security threats against VIPs: opportunities
and challenges\footnote{\emph{Bertrand De Longueville} --- DG HR,
  European Commission}}{Social Media Monitoring for Awareness of Security threats against VIPs: opportunities and challenges}}\label{social-media-monitoring-for-awareness-of-security-threats-against-vips-opportunities-and-challenges}

\begin{quote}
Abstract: Identifying possible adversaries is a key element of Security
Threat Assessments. When assessing threats against persons with high
public visibility, monitoring Social Media may seem a promising idea in
order to identify potential groups, persons, ideologies developing a
specific hate narrative. Indeed, social media features all sort of
opinion trends and are often used as an echo chamber for propaganda
purposes. They can thus be seen as an easily accessible and abundant
source for Personal Threat Assessment purposes. However, exploitation of
Social Media material into actionable intelligence (e.g.~to support
decisions on the set up of VIP security measures) poses several
methodological and technical challenges. The purpose of this
presentation is to stimulate discussion on such challenges, rather than
describing pre-cooked solutions.
\end{quote}

VIPs: - 28 European commissioners - any staff member exposed to high
visibility because of his/her functions - any staff member exposed to a
security threat because of his/her functions

80\%/90\% of false positives (for vocabulary search for the most violent
languages used)

\section[Computer support for analysing violent extremism in online
environments]{\texorpdfstring{Computer support for analysing violent
extremism in online environments\footnote{\emph{Magnus Sahlgren} magnus
  {[}dot{]} sahlgren {[}at{]} ri {[}dot{]} se --- FOI -- Swedish Defence
  Research Agency}}{Computer support for analysing violent extremism in online environments}}\label{computer-support-for-analysing-violent-extremism-in-online-environments}

\begin{quote}
Abstract: This talk gives an overview over the research done at the
Swedish Defense Research Agency (FOI) on developing and applying tools
for analysing violent extremism in online environments. The talk focuses
on analysis of text data, and presents some of the core technologies we
use to deal with large-scale and noisy data. We also provide examples
from two recent studies where we have applied our tools to large
collections of propaganda material from IS, and to large collections of
web data from Swedish right-wing extremist groups.
\end{quote}

Analyses: - theory driven (warnings defined by experts) - data driven

Main topics: 1. Lone wolf actors (eg, fixation, leakage) 2.
Radicalization (eg, in-group/out-group, dichotomous thinking)

\subsection{Linguistic marker:}\label{linguistic-marker}

\begin{enumerate}
\def\labelenumi{\arabic{enumi}.}
\tightlist
\item
  word list
\item
  vocabulary variation (synonyms)
\item
  Semantic memories
\end{enumerate}

\subsection{Thematic analyses:}\label{thematic-analyses}

Theme and word list (eg, BRUTALITY: exclude, punish, behead \ldots{})

Count occurrences of word in data Monitor theme over time

\subsection{Processing pipeline}\label{processing-pipeline}

segmentation - Language - Tokenization - Normalization

\begin{quote}
normalization improve recall but can reduce precision
\end{quote}

\begin{itemize}
\tightlist
\item
  overestimation problem: polysemous (eg, ``execute''): disambiguation
\item
  underestimation problem: synonymy (``IS/ISIS/ISIL/Daesh''): semantic
  memory
\end{itemize}

\subsection{End-to-end system}\label{end-to-end-system}

Character-based (deep) neural network Eliminate the need for
preprocessing, more accurate then lexical analyses (if trained with
sufficient data)

eg, thematic analyses of IS propaganda

\begin{itemize}
\tightlist
\item
  prior polarity list (sentiment analyses)
\end{itemize}

English: opinion lexicon, MPQA, sentiwordnet, LIWC, saifmohammed.com
(domain specific lexicon)

\begin{quote}
how to detect ironia?!?!
\end{quote}

Lexicon VS MLT

Annotation: - correlation as measure of reliability (\textgreater{}.8,
for someone \textgreater{}.9)

\begin{itemize}
\tightlist
\item
  demographic: gender, age, origin classification
\item
  author identification, alias matching
\item
  socio-political: education, \ldots{}
\end{itemize}

Analysis platform (not data collection!)

dark WEB (TOR): agora market (main drug market)

\section[Use of Open-Source Information in the IAEA Safeguards
Department]{\texorpdfstring{Use of Open-Source Information in the IAEA
Safeguards Department\footnote{\emph{Chris Eldrige} International Atomic
  Energy Agency}}{Use of Open-Source Information in the IAEA Safeguards Department}}\label{use-of-open-source-information-in-the-iaea-safeguards-department}

\begin{quote}
Abstract: The IAEA Department of Safeguards makes extensive use of
open-source information in support of its mission to verify the
compliance of Member States with their safeguards obligations.
Open-source information is one of several data streams that facilitate
the ongoing State evaluation process. This presentation will review the
department's work with open-source information, which includes both
routine monitoring of news and other information sources and targeted
searching, to support verification activities both in headquarters and
in the field. Sources used include websites, newsfeeds, scientific and
technical literature, and databases containing information on imports
and exports of commodities. The ongoing, productive relationship between
the IAEA and JRC Ispra has significantly strengthened the Department of
Safeguards' capabilities in routine monitoring of open sources, and the
IAEA is now exploring the possibility of using the EMM OSINT Suite to
streamline and improve the department's capability to perform targeted
searches of open sources for information of safeguards relevance.
\end{quote}

\begin{itemize}
\item
  credible assurance to the international community that States are
  honoring their safeguards obligations
\item
  correctness \& \emph{completeness}
\item
  \href{https://goo.gl/6rVxqt}{video}: inspecting the nuclear fuel cycle
\end{itemize}

\chapter{EMM workshop}\label{emm-workshop}

\begin{quote}
Overview: This course provides a general overview of EMM technology and
related tools. It shows with practical examples how Text Mining and
Analysis (TMA) can effectively support the daily work of analysts. The
platform processes every day about 300,000 news articles providing:
language detection, categorization, language recognition, entity
extraction, quote extraction, geotagging, tonality, duplicate detection,
categorisation, indexing and searching, clustering, statistics and event
extraction. Dedicated Graphical User Interfaces allow analysts to
display and browse all metadata and create reports and/or newsletters.
The course structure includes hands-on sessions based on a common use
case: participant will learn how to configure the platform in order to
capture news related to their topics of interest, browse the results,
produce newsletters and send notifications. This course is targeting
people who has just started using EMM or who need to assess whether to
adopt it as media monitoring platform.
\end{quote}

\emph{Big data Pilot Project}, \emph{Text and Data Mining unit} - EMM:
Europe Media Monitor - SITAF: Statistical and Information for Anti Fraud
TIM: Tools for Innovation Monitoring

\begin{quote}
for every people: do not store anything and do not need to be an expert
\end{quote}

\url{https://newsdesk.emm4u.eu} {[}osint2017{]}

\section{Aim: what EMM can and cannot
do.}\label{aim-what-emm-can-and-cannot-do.}

\begin{itemize}
\tightlist
\item
  have the platform on our server (they don't have no more resources to
  supply external request)
\end{itemize}

\begin{enumerate}
\def\labelenumi{\arabic{enumi}.}
\tightlist
\item
  Analyze unstructured text:
\end{enumerate}

\begin{itemize}
\tightlist
\item
  Natural language
\item
  Ambiguous/incomplete
\item
  Multi-language coverage
\end{itemize}

\begin{enumerate}
\def\labelenumi{\arabic{enumi}.}
\setcounter{enumi}{1}
\tightlist
\item
  \textasciitilde{}70 languages
\item
  DO NOT SCAN ANYTHING, it is not Google: it scan 70 source every 5
  minutes. Every source is as any of one which decide to collaborate to
  the project. If our interested source is not in EMM they cannot add
  it! They need a (open source) contract.
\item
  They provide statistics or insides on the text but if we want the real
  text we have to click on the link! (There are the email to WHO too)
\end{enumerate}

\section{What they do with the text:}\label{what-they-do-with-the-text}

\begin{enumerate}
\def\labelenumi{\arabic{enumi}.}
\tightlist
\item
  First thing is to asses the language! (detected with proprietary
  tools! 80\% accuracy overall)
\item
  Second, category: e.g. ``Taxation, Economy, EU-US trade'' --- 3k
  categories/topics, based on more 60k keywords. The quality of the
  results comes from the quality of the translation of the keywords (do
  not use Google translate!).
\end{enumerate}

\begin{quote}
we will do an exercise on this categorization process
\end{quote}

\begin{itemize}
\tightlist
\item
  once you have set the keywords of the topic EMM redo every day 24/7
  the categorization of sources on your topic!
\end{itemize}

\begin{enumerate}
\def\labelenumi{\arabic{enumi}.}
\setcounter{enumi}{3}
\item
  next, there are extracted the entities: people and organization. (for
  each entities they search for every variant of the name). 1.7M
  entities.
\item
  It extract also the quotes! from the entities. (note: the name of the
  street do not become entities {[}\ldots{}he said\ldots{}mmm{]})
\item
  Geo detection from the text, and add meta data with the lat and lon
\item
  sentiment analysis: positive or negative (based on positive and
  negatives keywords). Usually they do not show this results because
  taken by itself do not have a great meaning. on the other hand if we
  collect all the article of a resource over a year and we consider the
  average level of sentiment for that resource in that topic, then you
  can have a value that represent the trend of the sentiment of a
  specific paper into the subject in his context.it is also difficult to
  separate the sentiment of the people who wrote with the sentiment of
  the news or the topic (e.g.~a very good people that work in a very bad
  immigrant situation)
\end{enumerate}

\begin{quote}
Extract event metadata, only available for specific topic. Anyway it
only aims to reduce the ammount of work it is not magic.
\end{quote}

first: on the website we have to detect the correct part of text of
interest. 80\% of quality (1/5 of text retrieved are
fake/wrong/out-of-topic text)

8k news sites 300k article/day 70 languages 3k categories 60k keywords
runs 24/7 25k/day visitors

Domains: Border security, Cyber-crime and {[}?{]}

\section{Notes}\label{notes}

the system is \emph{live} so the new overwrite the older\ldots{} if you
are not on the screen you loose them

subscribe for topics: e.mail, rss, sms, \ldots{}: subscribe for a one
every 24 hours or even (do not did this) for a prompt alert for
something

translation of 20 languages in English. Why: because someone do not want
do tell to Google our keywords. there are only two security: to not
disclose the translation, the click, the search\ldots{} and the
copyright.

If you want to have the translation service provide by the EMM the have
to give you the servers (real..they are 4)

everything look at the interface is no-moderated!!!

Is it also possible to filter the article retrieved by \emph{the same
stories} but stories are language specific!

Multilingual aspect are very important. Even English do not cover all
the region of the word

you can look at the stories AFTER the selection on the topic in a
specific context

they can do past stories for the past but they do not have a graphical
interface for this. If we have a specific request (title of the article,
dates, etc\ldots{}) they can retrieve the information about its story.

\url{http://medisys.newsbrief.eu/medisys/homeedition/it/home.html}

\section{second service (not yet publicy) for
expert\ldots{}}\label{second-service-not-yet-publicy-for-expert}

\ldots{}in which you can define your research and store them

Use of twitter to extend the information not to create or compose them:
usually people read the news and next talk about the new. (newspaper
first!)

For twitter there are two level categorization, the first from a search
next a second screen based on a user search strings: top-ten user,
hashtags keywords and links (note: links are very important because the
content of the link are often out of the repository of the EMM resources
and so this list of links can inform on something before it appear in
the EMM)

\section{Main process-flow}\label{main-process-flow}

\begin{itemize}
\tightlist
\item
  Short document!!(1 page)
\end{itemize}

documenti entrano nel processo (dal web, dalle mail, \ldots{}) e
iniziano il processo: 1. identificazione della lingua 2. topic 3. geo
location 4. entities 5. quotations \ldots{}. \ldots{}

alla fine esce il blocco coi metadata e il testo e qui il testo viene
immediatamente cancellato! tutto il resto viene mantenuto per sempre,
inclusi il link, e tutti i metadata. e quindi possono essere interrogati
dal sistema come una sorta di google

\subsection{Exercice: create a
category}\label{exercice-create-a-category}

one file on topics based on keywords and one file on metadata

\section{Produce newsletter}\label{produce-newsletter}

Di fatto un account consiste in due file Alert e Filter. Il primo serve
a elencare e memorizzare le \emph{categorie} il secondo serve per fare
combinazioni opportune di categorie. Per esempio si possono combinare
categorie diverse, considerare solo determinate lingue, determinate
nazioni di origine, determinate risorse (source di informazioni da cui
pescano)

Main desks: \url{https://newsdesk.emm4u.eu/ND1/} CategoryEditor:
\url{https://newsdesk.emm4u.eu/CategoryEditor/AlertEditor.html\#}
WorkSpace: \url{https://newsdesk.emm4u.eu/ND1/?ws=true} EMMnewws:
\url{http://emm.newsbrief.eu/NewsBrief/sourceslist/it/list.html} Wiki;
\url{https://wiki.emm4u.eu/confluence/display/CE/Category+Definitions}
MediSys:
\url{http://medisys.newsbrief.eu/medisys/alertedition/en/Chagasdisease.html}

Per vedere le proprie categorie andare in una a caso dal EMMnews (ma
entrarci, fare attenzione che nell'indirizzo compaia ``alertedition'') e
sostituire la parte finale dell'indirizzo con il nome della propria
categoria. A questo punto ci si puo sottoscrivere alla categoria e
quindi si possono avere aggiornamenti giornalieri. (o istantanei, me
sconsigliatissimo!!), per esempio il nostro

\section{Contacts}\label{contacts}

charles {[}dot{]} macmillan {[}at{]} ec {[}dot{]} europa {[}dot{]} eu
jens {[}dot{]} linge {[}at{]} ec {[}dot{]} europa {[}dot{]} eu marco
{[}dot{]} verile {[}at{]} ec {[}dot{]} europa {[}dot{]} eu eleonora
{[}dot{]} mantica {[}at{]} ext {[}dot{]} ec {[}dot{]} europa {[}dot{]}
eu

\bibliography{book.bib,packages.bib}


\end{document}
